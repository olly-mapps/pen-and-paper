\documentclass[12pt, letterpaper]{article}

\usepackage{parskip}
\usepackage{tikz}
\usetikzlibrary{shapes.geometric}
\usepackage[shortlabels]{enumitem}

\begin{document}

\section*{Problem 2}
\textit{(INMO 1993)}

A convex hexagon is a hexagon in which none of its interior angles are more than $180$ degrees.

Show, by construction, that there exists a convex hexagon with the following properties

\begin{enumerate}[(a)]
\item All of its interior angle are equal
\item Its sides are length $1,2,3,4,5,6$ in some order

\end{enumerate}

\newpage

\section*{Solution}
To solve this problem methodically, we notice that since the interior angles should be equal, the hexagon should have some degree of regularity. 

In order to achieve this, we can use a regular shape as a stencil for our hexagon, say an equilateral triangle. This way we can, in a sense, inherit some of the regularity from the triangle

It seems sensible that the longest lengths ($4,5,6$) should lie on the border of the triangle, as these would be hard to fit inside the triangle as chords.

As it turns out, if we draw three more triangles within the main triangle of side lengths $1,2,3$, the sides add up nicely and we achieve our answer.

\begin{center}
\begin{tikzpicture}
\node[regular polygon, 
	regular polygon sides = 3,  draw,
	minimum width = 9cm, anchor = south] (t) at (0,0) {};
\node[anchor = north] at (t.south) {$6$};
\node[anchor = south, yshift = 0.5cm] at (t.east) {$4$};
\node[anchor = south, yshift = 0.5cm] at (t.west) {$5$};
	
\node[regular polygon, 
	regular polygon sides = 3, 
	draw,
	minimum width = 3cm, anchor = north] (three) at (t.north) {};
\node[anchor = north] at (three.south) {$3$};
	

\node[regular polygon, 
	regular polygon sides = 3, 
	draw,
	minimum width = 1cm, anchor = south] (one) at (-3.46, 0) {};
\node[anchor = south, xshift = 0.05cm] at (one.east) {$1$};
	
	
\node[regular polygon, 
	regular polygon sides = 3, 
	draw,
	minimum width = 2cm, anchor = south] (two) at (3.035, 0) {};
\node[anchor = south, yshift = 0.25cm] at (two.west) {$2$};
	
	
\end{tikzpicture}
\end{center}

\end{document}