\documentclass[12pt, letterpaper]{article}

\usepackage{parskip}
\usepackage{amsmath}
\usepackage{tikz}
\usetikzlibrary{shapes.geometric}
\usepackage[shortlabels]{enumitem}

\begin{document}

\section*{Problem 3}
\textit{(Putnam 2021)}

A grasshopper starts at the origin in the coordinate plane and makes a sequence of hops. Each
hop has length 5, and after each hop the grasshopper is at a point whose coordinates are both
integers; thus, there are 12 possible locations for the grasshopper after the first hop. What is
the smallest number of hops needed for the grasshopper to reach the point (2021, 2021)?


\newpage

\section*{Solution}

We first observe that any hop the grasshopper makes is a composition of x and y movements. Further, in order to minimise the number of hops taken, it seems sensible that each step should be non-negative in at least one of its components.

Let us list every sensible movement from the origin in coordinate form:

\[ \begin{pmatrix}5\\0\end{pmatrix}, \begin{pmatrix}0\\5\end{pmatrix},
\begin{pmatrix}3\\4\end{pmatrix},
\begin{pmatrix}-3\\4\end{pmatrix},
\begin{pmatrix}4\\3\end{pmatrix},
\begin{pmatrix}-4\\3\end{pmatrix}. \]


\end{document}